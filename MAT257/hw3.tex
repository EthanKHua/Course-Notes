\documentclass{article}
\usepackage[margin=1.0in]{geometry}
\usepackage{amssymb,amsmath,amsthm,amsfonts}
\usepackage{enumitem}
\usepackage{xcolor}
\usepackage{mathtools}

\newcommand{\Z}{\mathbf{Z}}
\newcommand{\N}{\mathbf{N}}
\newcommand{\R}{\mathbf{R}}
\newcommand{\Q}{\mathbf{Q}}
\newcommand{\C}{\mathbf{C}}

\newcommand{\id}{\mathrm{id}}
\newcommand{\op}{\mathrm{op}}
\newcommand{\diam}{\mathrm{diam}}
\newcommand{\Tr}{\mathrm{Tr}}
\newcommand{\im}{\mathrm{im}}
\newcommand{\rank}{\mathrm{rank}}

\newcommand{\cl}[1]{\overline{#1}}

\swapnumbers % places numbers before thm names

\theoremstyle{plain} % The "plain" style italicizes all body text.
	\newtheorem{thm}{Theorem}
		\numberwithin{thm}{section} % Theorem numbers are determined by section.
	\newtheorem{lemma}[thm]{Lemma}
	\newtheorem{prop}[thm]{Proposition}
	\newtheorem{cor}[thm]{Corollary}

\theoremstyle{definition}
    \newtheorem{defn}[thm]{Definition}
	\newtheorem{example}[thm]{Example}
	\newtheorem{exercise}[thm]{Exercise} %Exercise

\begin{document}
\stepcounter{section}
\stepcounter{section}
\section{Week 3 Homework}
    \exercise Let $S\subseteq C[0,1]$. Consider the following two statements:
    \begin{itemize}
        \item $S$ is an open subset of $(C[0,1],\|\cdot\|_1)$.
        \item $S$ is an open subset of $(C[0,1],\|\cdot\|_\infty)$.
    \end{itemize}
    Determine if the first statement implies the second, and vice-versa. Supply proof or counterexample as appropriate.

    \begin{proof}
        We claim that the first statement implies the second, but not the converse.

        Suppose that \(S\) is an open subset of \((C[0,1], \|\cdot\|_1)\). For any \(g \in S\), there is an open ball with respect to the 1-norm centered around \(g\) with radius \(\varepsilon\) such that \(B_1 (g, \varepsilon)\subseteq S\). We proceed to show that \(B_\infty (g, \varepsilon) \subseteq B_1 (g, \varepsilon)\). Let \(f \in B_\infty (g, \varepsilon)\). Then
        \[
            \|f-g\| _1 = \int _0^1 \left\vert f-g \right\vert \leq \int _0^1 \sup \{ \left\vert f-g \right\vert \} = \|f-g\| _\infty < \varepsilon 
        \]
        Thus \(B_\infty (g, \varepsilon) \subseteq B_1 (g, \varepsilon) \subseteq S\). Thus \(S\) is an open subset of \((C[0,1], \|\cdot \| _1)\).

        Now we show that the converse is not necessarily true. Let \(S = B_\infty (0, 1)\). This is an open subset of \((C[0,1],\|\cdot\|_\infty)\). Consider \(f(x) = 0 \in B_\infty (0, 1)\). For every \(\varepsilon > 0\), we can always find \(n \in \mathbb{N}\) such that \(n > \frac{1}{\varepsilon}\). Let \(g(x) = x^{n-1}\). Since \(\int _0^1 g(x)dx = \frac{1}{n} < \varepsilon\), \(g(x) \in B_1(0, \varepsilon)\). But \(g(1) = 1\), which means that \(g(x) \notin B_\infty (0, 1)\). Thus \(f\) is not an interior point of \(B_\infty (0, 1)\) with respect to the 1-norm, which means that \(B_\infty (0, 1)\) is not an open subset of \((C[0,1],\|\cdot\|_1)\).
        
    \end{proof}
    \exercise Let $X$ be any set. The \textbf{diagonal} of $X\times X$ is the following set:
    \[ \Delta = \{(x,x) : x \in X\}. \]
    Prove that if $(X,d)$ is a metric space, then $\Delta$ is a closed subset of $X\times X$ (with respect to the product metric).
    \exercise\textit{Can linear subspaces be open and/or closed?}

    \begin{enumerate}[label=(\alph*)]
        \item Let $C^\infty[0,1]$ denote the set of infinitely differentiably functions $f:[0,1]\rightarrow \R$. Prove that $C^\infty[0,1]$ is \textit{not} a closed subset of $(C[0,1],\|\cdot\|_\infty)$.

        \begin{proof}
            To show that this set is not closed, we just need to find a limit point that is not an element of the set. Let \(f(x) = |x-\frac{1}{2}|\). Clearly \(f \notin C^{\infty} [0,1]\). To prove \(f\) is a limit point of \(C^{\infty} [0,1]\), first let \(\varepsilon > 0\) and consider the open ball \(B(f, \varepsilon)\). There exists an \(n \in \mathbb{N}\) such that \(n + 1 > \frac{1}{\varepsilon} \implies asdfasfas\). 
            
        \end{proof}

        \item Let $C$ be the set of \textbf{convergent} sequences of real numbers. Prove that $C$ is a closed subset of $(\ell^\infty,\|\cdot\|_\infty)$.

        \item Let $(X,\|\cdot\|)$ be a normed vector space, and let $M$ be a \textcolor{red}{linear subspace} of $X$. Prove that $M$ is an open set if and only if $M=X$.
    \end{enumerate}

    \exercise \textit{The Bolzano--Weierstrass Theorem.}

    \begin{enumerate}[label=(\alph*)]
        \item Prove that every bounded sequence in $(\R^d,\|\cdot\|_2)$ has a convergent subsequence.
        
        \begin{proof}
            We take the Bolzano-Weierstrass Theorem in \(\mathbb{R}\) for granted and use this to prove it for \(\mathbb{R}^d\). We will do this using induction on \(d\). When \(d=1\), it follows trivially from the theorem in \(\mathbb{R}\).

            Now suppose that the claim is true for some \(d = k\), for some \(k \in \mathbb{N}\). Let \((a_n)_{n\geq1} \) be a bounded sequence in \((\R^k,\|\cdot\|_2)\). Define another sequence \((b_n)_{n\geq 1}\) in \(\mathbb{R}^{k-1}\) such that \(b_i\) is the first \(k-1\) components of \(a_i\). From our assumption, \(b_i\) has a convergent subsequence \((b_{n_i})_{i\geq1}\). Consider another sequence \((c_{n_i})_{i\geq0}\) in \(\mathbb{R}\) , where \(c_{n_i}\) is equal to the last component of \(a_{n_i}\). By the Theorem in \(\mathbb{R}\), \((c_{n_i})_{i\geq1}\) has a convergent subsequence \((c_{n_{m_i}})_{i\geq1}\). Suppose that \((b_{n_{m_i}})_{i\geq1}\) converges to \(B = (B_1, B_2, \dots, B_{k-1})\) and \((c_{n_{m_i}})_{i\geq1}\) converges to \(C\). We claim that \((a_{n_{m_i}})_{i\geq1}\) is our desired subsequence, which converges to \((B_1, B_2, \dots, B_{k-1}, C)\).

            Let \(\varepsilon > 0\). Since \((B_1, B_2, \dots, B_{k-1})\) and \((c_{n_{m_i}})_{i\geq1}\) converge,
            \[
                \exists N_b , N_c > 0 \text{ such that } n_b > N_b \implies \| b_{n_b} - B \| _2 < \frac{\varepsilon}{2}\text{ and } n_c > N_c \implies | c_{n_c} - C | < \frac{\varepsilon}{2}
            \]
            Let \(N = \max \{N_b, N_c\}\). Let \(n \in \mathbb{N}\), \(n > N\). Then
            \[
                \|a_n - (B_1, B_2, \dots, B_{k-1}, C) \| _2 = \sqrt{(a_1 - B_1)^2 + (a_2 - B_2)^2 + \dots + (a_{k-1} - B_{k-1})^2 + (a_k - C)^2} 
            \]
            Using an inequality that I don't know the name of, we have
            \[
                \sqrt{(a_1 - B_1)^2 + (a_2 - B_2)^2 + \dots + (a_{k-1} - B_{k-1})^2 + (a_k - C)^2}
            \]
            \[
                \leq \sqrt{(a_1 - B_1)^2 + (a_2 - B_2)^2 + \dots + (a_{k-1} - B_{k-1})^2} + \sqrt{(a_k - C)^2} = \| b_n - B \| _2 + | c_n - C | _2
            \]
            \[
                < \frac{\varepsilon}{2} + \frac{\varepsilon}{2} = \varepsilon
            \]
            Thus \((a_{n_{m_i}})_{i\geq1}\) converges, which means that \((a_n)_{n\geq1}\) does indeed have a convergent subsequence. By the principle of induction, the Bolzano-Weierstrass Theorem holds in \((\mathbb{R}^d, \|\cdot\|_2)\) and we are done.

        \end{proof}
        \item Give an example of a normed vector space $(X,\|\cdot\|)$ containing a \textcolor{red}{(bounded?)} sequence $(x_n)$ which has no convergent subsequences.
        
        \begin{proof}
            Consider the normed vector space \((\ell^{\infty}, \| \cdot \| _\infty)\). Let \((\vec{x_n})_{n\geq1}\) be a sequence in \((\ell^{\infty}, \| \cdot \| _\infty)\) defined by \(\vec{x_i}_k = \begin{dcases}
                1, &\text{ if } k=i ;\\
                0, &\text{ otherwise} .
            \end{dcases}\)

            Clearly \((\vec{x_n})_{n\geq1}\) is bounded. Let \((\vec{x_{n_i}})_{i\geq1}\) be a subsequence of \((\vec{x_n})_{n\geq1}\). We will show that \((\vec{x_{n_i}})_{i\geq1}\) diverges.

            For any \(L \in \ell^{\infty}\) Let \(\varepsilon = 1\). For any \(N \in \mathbb{N}\), we just take \(n = N\). Then
            \[
                \vec{x_i}_k
            \]
        \end{proof}
    \end{enumerate}
\end{document}