\documentclass{article}
\usepackage[margin=1.0in]{geometry}
\usepackage{amssymb,amsmath,amsthm,amsfonts}
\usepackage{enumitem}
\usepackage{xcolor}
\usepackage{mathtools}

\newcommand{\Z}{\mathbf{Z}}
\newcommand{\N}{\mathbf{N}}
\newcommand{\R}{\mathbf{R}}
\newcommand{\Q}{\mathbf{Q}}
\newcommand{\C}{\mathbf{C}}

\newcommand{\id}{\mathrm{id}}
\newcommand{\op}{\mathrm{op}}
\newcommand{\diam}{\mathrm{diam}}
\newcommand{\Tr}{\mathrm{Tr}}
\newcommand{\im}{\mathrm{im}}
\newcommand{\rank}{\mathrm{rank}}

\newcommand{\cl}[1]{\overline{#1}}

\swapnumbers % places numbers before thm names

\theoremstyle{plain} % The "plain" style italicizes all body text.
	\newtheorem{thm}{Theorem}
		\numberwithin{thm}{section} % Theorem numbers are determined by section.
	\newtheorem{lemma}[thm]{Lemma}
	\newtheorem{prop}[thm]{Proposition}
	\newtheorem{cor}[thm]{Corollary}

\theoremstyle{definition}
    \newtheorem{defn}[thm]{Definition}
	\newtheorem{example}[thm]{Example}
	\newtheorem{exercise}[thm]{Exercise} %Exercise

\begin{document}
\stepcounter{section}
\stepcounter{section}
\section{Week 3 Homework}
    \exercise Let $S\subseteq C[0,1]$. Consider the following two statements:
    \begin{itemize}
        \item $S$ is an open subset of $(C[0,1],\|\cdot\|_1)$.
        \item $S$ is an open subset of $(C[0,1],\|\cdot\|_\infty)$.
    \end{itemize}
    Determine if the first statement implies the second, and vice-versa. Supply proof or counterexample as appropriate.

    \begin{proof}
        We claim that the first statement implies the second, but not the converse.

        Suppose that \(S\) is an open subset of \((C[0,1], \|\cdot\|_1)\). For any \(g \in S\), there is an open ball with respect to the 1-norm centered around \(g\) with radius \(\varepsilon\) such that \(B_1 (g, \varepsilon)\subseteq S\). We proceed to show that \(B_\infty (g, \varepsilon) \subseteq B_1 (g, \varepsilon)\). Let \(f \in B_\infty (g, \varepsilon)\). Then
        \[
            \|f-g\| _1 = \int _0^1 \left\vert f-g \right\vert \leq \int _0^1 \sup \{ \left\vert f-g \right\vert \} = \|f-g\| _\infty < \varepsilon 
        \]
        Thus \(B_\infty (g, \varepsilon) \subseteq B_1 (g, \varepsilon) \subseteq S\). Thus \(S\) is an open subset of \((C[0,1], \|\cdot \| _1)\).

        Now we show that the converse is not necessarily true. Let \(S = B_\infty (0, 1)\). This is an open subset of \((C[0,1],\|\cdot\|_\infty)\). Consider \(f(x) = 0 \in B_\infty (0, 1)\). For every \(\varepsilon > 0\), we can always find \(n \in \mathbb{N}\) such that \(n > \frac{1}{\varepsilon}\). Let \(g(x) = x^{n-1}\). Since \(\int _0^1 g(x)dx = \frac{1}{n} < \varepsilon\), \(g(x) \in B_1(0, \varepsilon)\). But \(g(1) = 1\), which means that \(g(x) \notin B_\infty (0, 1)\). Thus \(f\) is not an interior point of \(B_\infty (0, 1)\) with respect to the 1-norm, which means that \(B_\infty (0, 1)\) is not an open subset of \((C[0,1],\|\cdot\|_1)\).
        
    \end{proof}
    \exercise\textit{Can linear subspaces be open and/or closed?}

    \begin{enumerate}[label=(\alph*)]
        \item Let $C^\infty[0,1]$ denote the set of infinitely differentiably functions $f:[0,1]\rightarrow \R$. Prove that $C^\infty[0,1]$ is \textit{not} a closed subset of $(C[0,1],\|\cdot\|_\infty)$.

        \begin{proof}
            To show that this set is not closed, we just need to find a limit point that is not an element of the set. Let \(f(x) = |x-\frac{1}{2}|\). Clearly \(f \notin C^{\infty} [0,1]\). To prove \(f\) is a limit point of \(C^{\infty} [0,1]\), first let \(\varepsilon > 0\) and consider the open ball \(B(f, \varepsilon)\). There exists an \(n \in \mathbb{N}\) such that \(n + 1 > \frac{1}{\varepsilon} \implies asdfasfas\). Let \[g_n(x) = \int _\frac{1}{2}^x \frac{e^{-\frac{n}{t}}-e^{-\frac{n}{1-t}}}{e^{-\frac{n}{t}}+e^{-\frac{n}{1-t}}} dt \text{, } x \in [0,1] \]

            Since the integrand is constructed using infinitely differentiable functions, \(g_n\) is infinitely differentiable as well, thus \(g_n \in C^{\infty}[0,1]\). Notice that \(g_n(1) = \int _\frac{1}{2}^1 \frac{e^{-\frac{n}{t}}-e^{-\frac{n}{1-t}}}{e^{-\frac{n}{t}}+e^{-\frac{n}{1-t}}} dt\). If we perform the substitution \(u = 1-t\), we see that
            \[
                g_n(1) = \int _\frac{1}{2}^1 \frac{e^{-\frac{n}{t}}-e^{-\frac{n}{1-t}}}{e^{-\frac{n}{t}}+e^{-\frac{n}{1-t}}} dt = -\int _\frac{1}{2}^0 \frac{e^{-\frac{n}{1-u}}-e^{-\frac{n}{u}}}{e^{-\frac{n}{1-u}}+e^{-\frac{n}{u}}} du = g_n(0)
            \]
            Thus we can conclude that
            \[
                g_n(1) = \frac{1}{2}(g_n(1) + g_n(0)) = \frac{1}{2}\left(\int _\frac{1}{2}^1 \frac{e^{-\frac{n}{t}}-e^{-\frac{n}{1-t}}}{e^{-\frac{n}{t}}+e^{-\frac{n}{1-t}}} dt + \int _\frac{1}{2}^0 \frac{e^{-\frac{n}{t}}-e^{-\frac{n}{1-t}}}{e^{-\frac{n}{1-t}}+e^{-\frac{n}{t}}} dt\right) = \frac{1}{2} \left(\right)
            \]
            which we can solve by using the substitution 
            
            We compute \(\|f-g_n\| _\infty = \sup \{|f(x) - g_n(x)| : x \in [0,1]\}\).

            Let \(h(x) = f(x) - g_n(x)\). We try to maximize \(|h(x)|\). Taking its derivative with respect to \(x\), we get
            \[
                h^\prime(x) = f^\prime(x) - g_n^\prime(x) = \frac{x-\frac{1}{2}}{|x-\frac{1}{2}|} - \frac{e^{-\frac{n}{x}}-e^{-\frac{n}{1-x}}}{e^{-\frac{n}{x}}+e^{-\frac{n}{1-x}}}
            \]
            There is a critical point at \(x = \frac{1}{2}\) since \(h^\prime(\frac{1}{2})\) is undefined. Otherwise, if \(x > \frac{1}{2}\),
            \[
                h^\prime(x) = 1 - \frac{e^{-\frac{n}{x}}-e^{-\frac{n}{1-x}}}{e^{-\frac{n}{x}}+e^{-\frac{n}{1-x}}} = \frac{2e^{-\frac{n}{1-x}}}{e^{-\frac{n}{x}}+e^{-\frac{n}{1-x}}} > 0
            \]
            If \(x < 0\),
            \[
                h^\prime(x) = -1 - \frac{e^{-\frac{n}{x}}-e^{-\frac{n}{1-x}}}{e^{-\frac{n}{x}}+e^{-\frac{n}{1-x}}} = \frac{-2e^{-\frac{n}{x}}}{e^{-\frac{n}{x}}+e^{-\frac{n}{1-x}}} < 0
            \]
            Checking all critical points and endpoints, we see that
            \[
                h(\frac{1}{2}) = 0
                h(0) = 
            \]
            
        \end{proof}

        \item Let $C$ be the set of \textbf{convergent} sequences of real numbers. Prove that $C$ is a closed subset of $(\ell^\infty,\|\cdot\|_\infty)$.

        \item Let $(X,\|\cdot\|)$ be a normed vector space, and let $M$ be a \textcolor{red}{linear subspace} of $X$. Prove that $M$ is an open set if and only if $M=X$.
    \end{enumerate}

    \exercise \textit{The Bolzano--Weierstrass Theorem.}

    \begin{enumerate}[label=(\alph*)]
        \item Prove that every bounded sequence in $(\R^d,\|\cdot\|_2)$ has a convergent subsequence.
        \item Give an example of a normed vector space $(X,\|\cdot\|)$ containing a sequence $(x_n)$ which has no convergent subsequences.
    \end{enumerate}
\end{document}