\documentclass{article}
\usepackage[margin=1.0in]{geometry}
\usepackage{amssymb,amsmath,amsthm,amsfonts}
\usepackage{enumitem}
\usepackage{xcolor}
\usepackage{marginnote} % put icons in margins

\newcommand{\Z}{\mathbf{Z}}
\newcommand{\N}{\mathbf{N}}
\newcommand{\R}{\mathbf{R}}
\newcommand{\Q}{\mathbf{Q}}
\newcommand{\C}{\mathbf{C}}
\newcommand{\op}{\mathrm{op}}
\newcommand{\cl}[1]{\overline{#1}}


\newcommand{\due}[1]{\reversemarginpar\marginnote
    {\textcolor{red}{#1}}
    }

\theoremstyle{definition}
	\newtheorem{exercise}{} %Exercise

\begin{document}

\noindent
    {\large \textbf{MAT257 -- Week 1 Homework - Ethan Hua}}

\exercise \due{9/13}
    Let $A_1,A_2,A_3,\ldots$ be a sequence of countable sets. Prove that $\bigcup_{i\geq 1} A_i$ is countable.

\begin{proof}
    Since \(A_i\) is countable, denote \(A_{ij}\) to be the \(j\)th element of set \(A_i\).

    Define \(f: \bigcup_{i \ge 1} A_i \to \mathbb{N}\) as 
    \[
        f(A_{ij})=2^{i}3^{j}
    \]
    We will show that \(f\) is injective. Let \(A_{pq}, A_{rs} \in  \bigcup_{i \geq 1} A_i\). Suppose that \(f(A_{pq} ) = f(A_{rs} )\). Then
    \[
        2^p 3^q = 2^r 3^s
    \]
    Since every integer has a unique prime factorisation, it follows that \(p = r\), \(q = s\). Thus \(A_{pq} = A_{rs} \).

    \noindent \\ Now, define the injection \(g: \mathbb{N} \to \bigcup_{i\geq 1} A_i\) to be \(g(n) = A_{1n} \). 
    
    \noindent Therefore by the Schröder–Bernstein theorem, \(|\bigcup_{i\geq 1} A_i | = |\mathbb{N} |\). Thus the set is countable.

\end{proof}
    
\exercise \due{9/13}
    Let $X,Y,Z$ be three vector spaces. Prove that $L^2(X,Y;Z)$ is isomorphic to $L(X,L(Y,Z))$.

   (For two vector spaces $X,Y$, we use $L(X,Y)$ to denote the space of linear mappings from $X$ to $Y$. For three vector spaces $X,Y,Z$, and a a function $\beta:X\times Y\rightarrow Z$, we let $\beta(\,\cdot\,,y_0)$ denote the function $X\rightarrow Z$ given by $x\mapsto \beta(x,y_0)$, where $y_0\in Y$ is some fixed vector; this is called the \textbf{$y_0$-slice} of $\beta$. The \textbf{$x_0$-slice} $\beta(x_0, \,\cdot\,)$ is defined similarly.

   \begin{proof}
        For a bilinear map \(T \in L^2(X,Y;Z)\), \((x,y) \in X \times Y\), Define \(\phi : L^2(X,Y;Z) \to L(X,L(Y,Z))\) such that
        \[
            \phi (T)(x,y) = T(x, \cdot)(y)
        \]  
        where \(T(x,\cdot)\) is the \(x\)-slice of \(T\). We claim that this transformation is an isomorphism.

        First, let \(\phi (T) = 0\). Then \(\forall x \in X \text{,} T(x, \cdot)(y) = 0\), from which it follows that \(T(x, y) = 0\), meaning \(\phi\) is injective. 
        
        Next, fix \(U \in L(X,L(Y,Z))\). Let \(\beta \) be the basis for \(X\). Let \(T \in L^2(X,Y;Z)\) be the linear transformation such that \(T(x, \cdot) = U(x) \text{, for all } x \in \beta \). We see that \(\forall x,y \in X \times Y\),
        \[
            \phi (T)(x,y) = T(x, \cdot)(y) = U(x)(y)
        \]
        making \(\phi \) surjective.

        Thus \(\phi \) is an isomorphism, and we get that \(L^2(X,Y;Z) \cong L(X,L(Y,Z))\) as desired.
   \end{proof}

\exercise \due{9/13}
    Let $I=(a,b)$ and $J=(c,d)$ be two open intervals on the real line. Let $f:I\rightarrow J$ be an increasing function such that $f(I)$ is dense in $J$. Prove that $f$ is continuous.

    (For two sets $D,S\subseteq \R$ we say that $D$ is \textbf{dense} in $S$ if $D\cap (s-\varepsilon,s+\varepsilon)\neq\varnothing$ for all $s\in S$ and all $\varepsilon>0$.)
\begin{proof}
    Fixing an \(a \in I\), let \(\varepsilon > 0\). We can assume without loss of generality that \(\varepsilon \) is small enough that \((f(a) - \varepsilon , f(a) + \varepsilon ) \subseteq J\). Since \(f(I)\) is dense in \(J\), we can always find an \(y_1 \in (f(a) - \varepsilon , f(a)) \text{ and } y_2 \in (f(a), f(a) + \varepsilon )\) such that \(y_1 , y_2 \in f(I)\), which means \(y_1 = f(x_1) \text{ and } y_2 = f(x_2)\) for some \(x_1, x_2 \in I\).

    Take \(\delta = \min \{\lvert a - x_1 \rvert , \lvert a - x_2 \rvert \}\). Let \(x \in I\). Suppose that \(\lvert x - a \rvert\leq \delta \). If \(x = a\), clearly \(\lvert f(x) - f(a) \rvert < \varepsilon \). Consider when \(x < a\).

    We see that due to the choice of \(\delta \), we have that \(x_1 < x < a\). Using the fact that \(f\) is increasing, we obtain
    \[
        f(a) - \varepsilon < y_1 = f(x_1) < f(x) < f(a) \implies - \varepsilon < f(x) - f(a) < 0 \implies f(a) - f(x) = \lvert f(x) - f(a) \rvert < \varepsilon 
    \] 
    The argument for the case when \(x < a\) is almost the exact same, except for the use of \(x_2 \text{ and } y_2\) instead of \(x_1\) and \(y_1\), as well as the inequalities being swapped.

    With this, we can conclude that \(f\) is continuous.

\end{proof}
\end{document}