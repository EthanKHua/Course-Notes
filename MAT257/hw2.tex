\documentclass{article}
\usepackage[margin=1.0in]{geometry}
\usepackage{amssymb,amsmath,amsthm,amsfonts}
\usepackage{enumitem}
\usepackage{xcolor}
\usepackage{mathtools}

\newcommand{\Z}{\mathbf{Z}}
\newcommand{\N}{\mathbf{N}}
\newcommand{\R}{\mathbf{R}}
\newcommand{\Q}{\mathbf{Q}}
\newcommand{\C}{\mathbf{C}}

\newcommand{\id}{\mathrm{id}}
\newcommand{\op}{\mathrm{op}}
\newcommand{\diam}{\mathrm{diam}}
\newcommand{\Tr}{\mathrm{Tr}}
\newcommand{\im}{\mathrm{im}}
\newcommand{\rank}{\mathrm{rank}}

\newcommand{\cl}[1]{\overline{#1}}

\swapnumbers % places numbers before thm names

\theoremstyle{plain} % The "plain" style italicizes all body text.
	\newtheorem{thm}{Theorem}
		\numberwithin{thm}{section} % Theorem numbers are determined by section.
	\newtheorem{lemma}[thm]{Lemma}
	\newtheorem{prop}[thm]{Proposition}
	\newtheorem{cor}[thm]{Corollary}

\theoremstyle{definition}
    \newtheorem{defn}[thm]{Definition}
	\newtheorem{example}[thm]{Example}
	\newtheorem{exercise}[thm]{Exercise} %Exercise

\begin{document}
    \section{Week 2 Homework - Ethan Hua}
    \begin{exercise}
        \begin{enumerate}[label=(\alph*)]
            \item Prove that there exists an infinitely differentiable function $\alpha:\R\rightarrow \R$ such that $\alpha(t)=0$ for all $t\leq 0$, and $\alpha(t)>0$ for all $t>0$.
    
            \begin{proof}
                We define \(\alpha (t) =
                \begin{dcases}
                    0, &\text{ if } t \leq 0 ;\\
                    e^{-\frac{1}{t}} , &\text{ if } t > 0.\\
                \end{dcases}\) 
                
                Trivially \(\alpha (t) = 0 \text{ if } t \leq 0\) and \(\alpha (t) > 0 \text{ if }t > 0\). It remains to show that \(\alpha \) is infinitely differentiable.

                Since \(0\) is infinitely differentiable, and \(e^{-\frac{1}{x}}\) is infinitely differentiable for \(x > 0\), it suffices to show that derivatives of all orders of \(\alpha \) are continuous at \(t = 0\).

                We will prove using induction that
                \[
                    \forall n \in \mathbb{N} \cup \{0\} \text{, } \lim_{t \to 0} \alpha ^{(n)} (t) = 0
                \]

                First, we will show that 
                We will only worry about the right hand limit, as the left hand limit always evaluates to \(0\).

                For \(n = 0\),
                \[
                    \lim_{t \to 0^+} \alpha ^{(0)} (t) = \lim_{t \to 0^+} \alpha (t) = \lim_{t \to 0^+} e^{-\frac{1}{x}} = 0
                \]
                Thus the case for \(n = 0\) holds.

            \end{proof}
    
            \item Prove that there exists an infinitely differentiable function $\beta:\R\rightarrow \R$ such that $\beta(t)=1$ for all $t\geq 1$, and $\beta(t)=0$ for all $t\leq 0$.
    
            \textcolor{blue}{Hint: The shape you're looking for is $\dfrac{X}{X+Y}$.}
    
            \item Prove that there exists an infinitely differentiable function $\varphi:\R\rightarrow \R$ such that $\varphi(t)=1$ for all $t\in [2,3]$, and $\varphi(t)=0$ for $t\in \R\setminus (1,4)$.
    
            \textcolor{blue}{Hint: Your function $\beta(t)$ does half the job. Make a function $\gamma(t)$ that does the other half of the job. Then multiply them together.}
        \end{enumerate}
    \end{exercise}
    \begin{exercise}
        Let $S\subseteq \R^n$. Consider the following three statements:
        \begin{itemize}
            \item $S$ is a bounded subset of $(\R^n,\|\cdot\|_1)$.
            \item $S$ is a bounded subset of $(\R^n,\|\cdot\|_2)$.
            \item $S$ is a bounded subset of $(\R^n,\|\cdot\|_{\max})$.
        \end{itemize}
        Among these statements, determine which implications are true and which are false. There are six implications to investigate. Supply proof or counterexample as appropriate. Include pictures.
        \\\\
        \noindent\textbf{Claim.} Denote each statement as (1), (2), and (3), respectively. We claim that each statement implies all the other statements.
        \begin{proof}
            First of all, I am so sorry that you have to read this. Secondly, in this proof, we denote:
            \begin{enumerate}
                \item an open ball with respect to the 1-norm as \(B_1(p, \varepsilon )\)
                \item an open ball with respect to the Euclidean norm as \(B_2(p, \varepsilon )\) 
                \item an open ball with respect to the max-norm as \(B_{\max} (p, \varepsilon )\)  
            \end{enumerate}
            \((1) \implies (2)\):

            Suppose (1) is true. There exists a \(p = (p_1, p_2, \dots, p_n) \in \mathbb{R}^n\) and \(r > 0\) such that
            \[
                S \subseteq B_1(p, r)
            \] 
            We will show that \(S \subseteq B_2(p, r)\). Let \(x = (x_1, x_2, \dots, x_n) \in S\). Since \(S \subseteq B_1(p, r)\), by the triangle inequality,
            \[
                \left\lVert x - p \right\rVert _2 = \sqrt{\sum_{i=1}^{n} (x_i - p_i)^2} \leq \sum_{i=0}^n \left\vert x_i - p_i \right\vert  = \left\lVert x - p \right\rVert _1 < r
            \]
            Thus \(x \in B_2(p, r)\).

            \noindent\((2) \implies (1)\):
            
            We will show using induction on \(n\) that if for some open ball in \(\mathbb{R} ^n\) , \(S \subseteq B_2(p, r)\),  then \(S \subseteq B_1(p, \sqrt{n} r )\) 

            Let \(n = 1\). Suppose (2) is true. Fix \(x \in S \subseteq \mathbb{R} \). Then by (2),
            \[
                r > \left\lVert x - p \right\rVert _2 = \sqrt{(x - p)^2} = \left\vert x - p \right\vert = \left\lVert x - p \right\rVert _1 \implies S \subseteq B_1(p, \sqrt{1} \cdot r) 
            \]
            Thus the claim holds for \(n = 1\). Now suppose the claim holds for \(n = k\). Assume (2) is true. Let \(x \in S\). By (2),
            \[
                \left\lVert x - p \right\rVert _2 = \sqrt{\sum_{i=1}^{k+1} (x_i - p_i)^2} < r \implies \sum_{i=1}^{k+1} (x_i - p_i)^2 < r^2\tag{*}
            \]
            We want to show that \(x \in B_1(p, \sqrt{n} r)\), or
            \[
                \left\lVert x - p \right\rVert _1 = \sum_{i=0}^{k+1} \left\vert x_i - p_i \right\vert < \sqrt{n} r \iff \left(\sum_{i=0}^{k+1} \left\vert x_i - p_i \right\vert\right)^2 < nr^2 
            \]
            We have
            \[
                \left(\sum_{i=0}^{k+1} \left\vert x_i - p_i \right\vert\right)^2 = \left(\sum_{i=0}^{k} \left\vert x_i - p_i \right\vert + \left\vert x_{k+1} - p_{k+1} \right\vert \right)^2
            \]
            \[
                = \left(\sum_{i=0}^{k} \left\vert x_i - p_i \right\vert\right)^2 + 2\left\vert x_{k+1} - p_{k+1} \right\vert \left(\sum_{i=0}^{k} \left\vert x_i - p_i \right\vert\right) + \left\vert x_{k+1} - p_{k+1} \right\vert ^2 \tag{**}
            \]
            Manipulating the inequality from (*), we obtain
            \[
                \sum_{i=1}^{k} (x_i - p_i)^2 < r^2 - \left\vert x_{k+1} - p_{k+1}  \right\vert ^2
            \]
            which implies that \((x_1, x_2, \dots , x_k)\) is in the open ball \(B_2(p, r)\) on \(\mathbb{R} ^k\). From the indudction hypothesis,
            \[
                (**) < k(r^2 - \left\vert x_{k+1} - p_{k+1}  \right\vert ^2) + 2\left\vert x_{k+1} - p_{k+1} \right\vert \left(\sum_{i=0}^{k} \left\vert x_i - p_i \right\vert\right) + r^2 - \sum_{i=1}^{k} (x_i - p_i)^2
            \]
            \[
                = (k+1)r^2 - \left(k\left\vert x_{k+1} - p_{k+1}  \right\vert ^2 - 2\left\vert x_{k+1} - p_{k+1} \right\vert \left(\sum_{i=0}^{k} \left\vert x_i - p_i \right\vert\right) + \sum_{i=1}^{k} (x_i - p_i)^2\right)
            \]
            \[
                = (k+1)r^2 - \sum_{i=1}^k \left(\left\vert x_{k+1} - p_{k+1}  \right\vert ^2 - 2\left\vert x_{k+1} - p_{k+1} \right\vert\left\vert x_i - p_i \right\vert + (x_i - p_i)^2\right)
            \]
            \[
                = (k+1)r^2 - \sum_{i=1}^k \left(\left\vert x_{k+1} - p_{k+1}  \right\vert - \left\vert x_i - p_i \right\vert \right) ^2 \leq (k+1)r^2
            \]
            In summary, we have that
            \[
                \left\lVert x - p \right\rVert _1 < \sqrt{k+1} r
            \]
            Thus \(x\) is in the open ball \(B_1(p, \sqrt{k+1} r)\), which means that (1) is true. By the principle of induction, the claim holds for all \(n \in \mathbb{N} \). 

            \noindent\((1) \implies (3)\):
            Suppose (1). Then \(S\) is a subset of some open ball \(B_1(p, r)\). We will show that \(S \subseteq B_{max} (p, r)\).
            
            Let \(x \in S\). Then by (1):
            \[
                \left\lVert x - p \right\rVert _{\max} = \max_{1 \leq i \leq n} \{\left\vert x_i - p_i \right\vert \} \leq \sum_{i=1} ^n \left\vert x_i - p_i \right\vert = \left\lVert x - p \right\rVert _1 < r
            \]
            Thus \(x \in B_{\max} (p, r)\), so (3) is true.
            
            \noindent \((3) \implies (1)\):

            Suppose that (3) holds true. Then for some open ball \(B_{\max}(p, r)\). We will show that \(S \subseteq B_1(p, nr)\).
            
            Let \(x \in S\). We have
            \[
                \left\lVert x - p \right\rVert _{1} = \sum_{i=1} ^n \left\vert x_i - p_i \right\vert \leq n\max_{0 \leq i \leq n} {\left\vert x_i - p_i \right\vert } = n \left\lVert x - p \right\rVert _{\max} < nr
            \]
            Thus \(x \in B_1(p, nr)\), which implies that (1) is true.

            \noindent \((2) \implies (3)\):
            
            Suppose (2) is true. Then \(S \subseteq B_2(p,r)\), where \(B_2(p, r)\) is some open ball on the Euclidean norm. We want to show that \(S \subseteq B_{\max }(p, r)\). Indeed, we have
            \[
                \left\lVert x - p \right\rVert _{\max } = \max _{1 \leq i \leq n} \left\vert x_i - p_i \right\vert \leq \sqrt{\sum_{i=1} ^n \left\vert x_i - p_i \right\vert ^2} < r 
            \] 
            Thus \(S \subseteq B_{\max}(p, r)\), so (3) is true.
            
            \noindent \((3) \implies (2)\):
            
            Suppose that (3) holds true. Then for some open ball with the max-norm, \(S \subseteq B_{\max}(p, r)\). We will show that \(S \subseteq B_2(p, nr)\)
            
            Let \(x \in S\). Then be the triangle inequality,
            \[
                \left\lVert x - p \right\rVert _2 = \sqrt{\sum_{i = 1}^n \left\vert a_i - p_i \right\vert } \leq \sum_{i=1}^n \left\vert a_i - p_i \right\vert \leq n\max_{1 \leq i \leq n} \left\vert x_i - p_i \right\vert < n \left\lVert x - p \right\rVert _{\max} < nr
            \]
            Thus \(x \in B_2(p,nr)\), so (2) holds.
            
            \noindent Therefore, we have proven every implication to be true.

        \end{proof}
    \end{exercise}
    \begin{exercise}
        Let $(X,\|\cdot\|_X)$ and $(Y,\|\cdot\|_Y)$ be two normed vector spaces. A linear mapping $T:X\rightarrow Y$ is called \textbf{bounded} if there exists a constant $M\geq 0$ such that
        \[ \|T(x)\|_Y \leq M \|x\|_X \quad \text{for all $x\in X$.}  \]
        Let $B(X,Y)$ denote the set of these bounded linear operators. The \textbf{operator norm} on $B(X,Y)$, denoted by $\|\cdot\|_{\mathrm{op}}$, is defined as follows:
            \[ \|T\|_{\mathrm{op}} = \sup\{ \|T(x)\|_Y : x\in X \text{ and } \|x\|_X\leq 1 \}. \]
        \begin{enumerate}[label=(\alph*)]
            \item Prove that $B(X,Y)$ is a linear subspace of $L(X,Y)$.
            
            \begin{proof}
                The 0-transformation \(\left\lVert Z(x) \right\rVert  = 0 \leq \left\lVert x \right\rVert _X \text{, } \forall x \in X\), because of the definition of a norm. Thus \(0 \in B(X,Y)\).
                Let \(T, U \in B(X,Y) \text{, } c \in \mathbb{R}\). Then for all \(x \in X\), there exist \(M, N \geq 0\) such that
                \[
                    T(x) \leq M \left\lVert x \right\rVert _X \text{ and } U(x) \leq N \left\lVert x \right\rVert _X \implies T(x) + U(x) \leq (M+N) \left\lVert x \right\rVert _X
                \] 
                Which implies that \(T+U\) is a member of \(B(X,Y)\). As well, from the first inequality,
                \[
                    T(x) \leq M \left\lVert x \right\rVert _X \implies cT(x) \leq  cM \left\lVert x \right\rVert _X
                \]
                Which implies that \(cT\) is a member of \(B(X,Y)\).

                Since \(0 \in B(X,Y)\)and \(B(X,Y)\) is closed under addition and scalar multiplication, \(B(X,Y)\) is a subspace of \(L(X,Y)\).

            \end{proof}
            \item Prove that $\|\cdot\|_{\mathrm{op}}$ is a norm on $B(X,Y)$.
        
            \begin{proof}
                To prove that the operator norm is a norm, we first verify that \(\left\lVert T \right\rVert _{op} = 0 \iff T = 0\).

                Fix \(T \in B(X,Y)\) and suppose that \(T = 0\). Then for all \(x \in X \text{, } \left\lVert x \right\rVert _X \leq 1 \text{, } \left\lVert T(x) \right\rVert _Y = \left\lVert 0 \right\rVert _Y = 0\). Thus \(\left\lVert T \right\rVert _{op} = 0\).  
                
                Now suppose the converse, that \(\left\lVert T \right\rVert _{op} = 0\). Then 
                \[
                    \forall x \in X, \left\lVert x \right\rVert _X \leq 1, \left\lVert T(x) \right\rVert _Y \leq 0.
                \]
                But by the definition of the norm in Y,
                \[
                    0 \leq \left\lVert T(x) \right\rVert _Y
                \]
                It follows that \(T(x) = 0\).

                We denote the set of elements \(x\) in \(X\) such that \(\left\lVert x \right\rVert _X \leq 1\) as \(X^\prime\).

                To show nonnegativity, we note that for \(x \in X^\prime\),
                \[
                    \left\lVert T \right\rVert _{op} \geq \left\lVert T(x) \right\rVert _Y \geq 0
                \]

                To show homogeity, let \(T \in G(X,Y), c \in \mathbb{R}\). Then
                \[
                    \left\lVert cT \right\rVert _{op} = \sup \{ \left\lVert cT(x) \right\rVert _Y : x \in X^\prime\} = \sup \{ c\left\lVert T(x) \right\rVert _Y : x \in X^\prime\} = c\sup \{ \left\lVert T(x) \right\rVert _Y : x \in X^\prime\} = c \left\lVert T \right\rVert _{op}
                \] 

                Now we show that the triangle inequality holds with respect to the operator norm.

                Fix \(T, U \in B(X,Y)\). We denote the set of elements \(x\) in \(X\) such that \(\left\lVert x \right\rVert _X \leq 1\) as \(X^\prime\).
                Let \(x \in X^\prime\). By definition,
                \[
                    T(x) \leq \sup T(X^\prime) \text{ and } U(x) \leq \sup U(X^\prime)
                \]
                Adding both together obtains
                \[
                    T(x) + U(x) \leq \sup T(X^\prime) + \sup U(X^\prime)
                \]
                We see that \(\sup T(X^\prime) + \sup U(X^\prime)\) is an upper bound for \(T(x) + U(x)\). By the definition of the least upper bound,
                \[
                    \sup \{T(X^\prime)+U(X^\prime)\} \leq  \sup T(X^\prime) + \sup U(X^\prime) \implies \left\lVert T+U \right\rVert _{op} \leq \left\lVert T \right\rVert _{op} + \left\lVert U \right\rVert _{op}
                \]
                Thus the operator norm is, indeed, a norm.

            \end{proof}
            \item Let $T:\R^2\rightarrow \R^2$ be the linear mapping given by $T(x,y)=(x+y,x)$. Find, with proof, the exact value of $\|T\|_{\mathrm{op}}$. (Here, $\R^2$ is equipped with the usual norm.)
        
            \item Find, with proof, an example of an unbounded linear operator.
            
            \begin{proof}
                Define \(\ell ^0\) to be the set of all sequences that are eventually 0. Consider the metric spaces \((\ell^{0}, \left\lVert \cdot \right\rVert _{\ell \infty})\) and \((C[0, 1], \left\lVert \cdot \right\rVert _{C\infty} )\). Here, we denote \(\left\lVert \cdot \right\rVert _{\ell \infty}\) as the sup norm on \(\ell ^{\infty} \) and \(\left\lVert \cdot \right\rVert _{C\infty}\) as the sup norm on \(C[0,1]\)..
                
                Let \(T: \ell^{0} \to C[0,1]\) be defined by
                \[
                    T((a_n)_n) = \sum_{i=0}^k a_i i^x \text{, where } k \text{ is the last index where } a_k \neq 0
                \]
                First, we will show that \(T\) is a linear transformation. Fix \((a_n)_n, (b_n)_n \in \ell ^0, c \in \mathbb{R}\). Let \(k = \max \{ k_a, k_b \} \), where \(k_a, k_b\) are the last index where \(a_{k_a}\) and \(b_{k_b}\) are non-zero, respectively. Then
                \[
                    T(c(a_n)+(b_n)) = \sum_{i=0}^k (ca_i + b_i) i^x = c\sum_{i=0}^k a_i i^x + \sum_{i=0}^k b_i i^x = c\sum_{i=0}^{k_a} a_i i^x + \sum_{i=0}^{k_b} b_i i^x = cT((a_n)) + T((b_n))
                \]
                This verifies that \(T\) is a linear transformation.

                Now we show that \(T\) is unbounded. Fix \(M \geq 0\). Let \((a_n)_n \in \ell ^0\) such that \(a_i = 1\) if \(i=M+1\) and 0 otherwise. We have that
                \[
                    \left\lVert T((a_n)) \right\rVert _{C\infty} = \left\lVert (M+1)^x \right\rVert _{C\infty} = M+1 > M = M \left\lVert (a_n) \right\rVert _{\ell \infty}
                \]
                Thus \(T\) is an unbounded linear operator.

            \end{proof}
        \end{enumerate}
    \end{exercise}
\end{document}