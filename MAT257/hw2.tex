\documentclass{article}
\usepackage[margin=1.0in]{geometry}
\usepackage{amssymb,amsmath,amsthm,amsfonts}
\usepackage{enumitem}
\usepackage{xcolor}
\usepackage{mathtools}

\newcommand{\Z}{\mathbf{Z}}
\newcommand{\N}{\mathbf{N}}
\newcommand{\R}{\mathbf{R}}
\newcommand{\Q}{\mathbf{Q}}
\newcommand{\C}{\mathbf{C}}

\newcommand{\id}{\mathrm{id}}
\newcommand{\op}{\mathrm{op}}
\newcommand{\diam}{\mathrm{diam}}
\newcommand{\Tr}{\mathrm{Tr}}
\newcommand{\im}{\mathrm{im}}
\newcommand{\rank}{\mathrm{rank}}

\newcommand{\cl}[1]{\overline{#1}}

\swapnumbers % places numbers before thm names

\theoremstyle{plain} % The "plain" style italicizes all body text.
	\newtheorem{thm}{Theorem}
		\numberwithin{thm}{section} % Theorem numbers are determined by section.
	\newtheorem{lemma}[thm]{Lemma}
	\newtheorem{prop}[thm]{Proposition}
	\newtheorem{cor}[thm]{Corollary}

\theoremstyle{definition}
    \newtheorem{defn}[thm]{Definition}
	\newtheorem{example}[thm]{Example}
	\newtheorem{exercise}[thm]{Exercise} %Exercise

\begin{document}
    \section{Week 2 Homework - Ethan Hua}
    \begin{exercise}
        \begin{enumerate}[label=(\alph*)]
            \item Prove that there exists an infinitely differentiable function $\alpha:\R\rightarrow \R$ such that $\alpha(t)=0$ for all $t\leq 0$, and $\alpha(t)>0$ for all $t>0$.
    
            \begin{proof}
                We define \(\alpha (t) =
                \begin{dcases}
                    0, &\text{ if } t \leq 0 ;\\
                    e^{-\frac{1}{t}} , &\text{ if } t > 0.\\
                \end{dcases}\) 
                
                Trivially \(\alpha (t) = 0 \text{ if } t \leq 0\) and \(\alpha (t) > 0 \text{ if }t > 0\). It remains to show that \(\alpha \) is infinitely differentiable.

                Since \(0\) is infinitely differentiable, and \(e^{-\frac{1}{x}}\) is infinitely differentiable for \(x > 0\), it suffices to show that derivatives of all orders of \(\alpha \) are continuous at \(t = 0\).

                We will prove using induction that
                \[
                    \forall n \in \mathbb{N} \cup \{0\} \text{, } \lim_{t \to 0} \alpha ^{(n)} (t) = 0
                \]
                We will only worry about the right hand limit, as the left hand limit always evaluates to \(0\).

                For \(n = 0\),
                \[
                    \lim_{t \to 0^+} \alpha ^{(0)} (t) = \lim_{t \to 0^+} \alpha (t) = \lim_{t \to 0^+} e^{-\frac{1}{x}} = 0
                \]
                Thus the case for \(n = 0\) holds.

                Now suppose that the claim holds for \(n = k,\) for some \(k \in \mathbb{N} \cup \{0\}\). It can be shown that when \(t > 0\), 
                \[
                    \alpha ^{(k)} (t) = \sum_{i=0}^{\infty} \frac{(-1)^{i+k+1}(i+k+1)!}{t^{i+k+1} i!(i+1)!} = \frac{(-1)^{k+1}(k+1)!}{t^{k+1}} + \sum_{i=1}^{\infty} \frac{(-1)^{i+k+1}(i+k+1)!}{t^{i+k+1} i!(i+1)!}
                \]

            \end{proof}
    
            \item Prove that there exists an infinitely differentiable function $\beta:\R\rightarrow \R$ such that $\beta(t)=1$ for all $t\geq 1$, and $\beta(t)=0$ for all $t\leq 0$.
    
            \textcolor{blue}{Hint: The shape you're looking for is $\dfrac{X}{X+Y}$.}
    
            \item Prove that there exists an infinitely differentiable function $\varphi:\R\rightarrow \R$ such that $\varphi(t)=1$ for all $t\in [2,3]$, and $\varphi(t)=0$ for $t\in \R\setminus (1,4)$.
    
            \textcolor{blue}{Hint: Your function $\beta(t)$ does half the job. Make a function $\gamma(t)$ that does the other half of the job. Then multiply them together.}
        \end{enumerate}
    \end{exercise}
    \begin{exercise}
        Let $S\subseteq \R^n$. Consider the following three statements:
        \begin{itemize}
            \item $S$ is a bounded subset of $(\R^n,\|\cdot\|_1)$.
            \item $S$ is a bounded subset of $(\R^n,\|\cdot\|_2)$.
            \item $S$ is a bounded subset of $(\R^n,\|\cdot\|_{\max})$.
        \end{itemize}
        Among these statements, determine which implications are true and which are false. There are six implications to investigate. Supply proof or counterexample as appropriate. Include pictures.
    \end{exercise}
    \begin{exercise}
        Let $(X,\|\cdot\|_X)$ and $(Y,\|\cdot\|_Y)$ be two normed vector spaces. A linear mapping $T:X\rightarrow Y$ is called \textbf{bounded} if there exists a constant $M\geq 0$ such that
        \[ \|T(x)\|_Y \leq M \|x\|_X \quad \text{for all $x\in X$.}  \]
        Let $B(X,Y)$ denote the set of these bounded linear operators. The \textbf{operator norm} on $B(X,Y)$, denoted by $\|\cdot\|_{\mathrm{op}}$, is defined as follows:
            \[ \|T\|_{\mathrm{op}} = \sup\{ \|T(x)\|_Y : x\in X \text{ and } \|x\|_X\leq 1 \}. \]
        \begin{enumerate}[label=(\alph*)]
            \item Prove that $B(X,Y)$ is a linear subspace of $L(X,Y)$.
        
            (In MAT257, the term ``linear subspace'' means what ``subspace'' meant in MAT240, \textit{i.e.} ``a nonempty subset of a vector space which is closed under addition and scalar multiplication.'')
            
            \item Prove that $\|\cdot\|_{\mathrm{op}}$ is a norm on $B(X,Y)$.
        
            \item Let $T:\R^2\rightarrow \R^2$ be the linear mapping given by $T(x,y)=(x+y,x)$. Find, with proof, the exact value of $\|T\|_{\mathrm{op}}$. (Here, $\R^2$ is equipped with the usual norm.)
        
            \item Find, with proof, an example of an unbounded linear operator.
        \end{enumerate}
    \end{exercise}
\end{document}