\documentclass[11pt]{article}
\usepackage{amsmath,amssymb,amsthm,enumerate,nicefrac,fancyhdr,hyperref,graphicx,adjustbox,mathtools}
\hypersetup{colorlinks=true,urlcolor=blue,citecolor=blue,linkcolor=blue}
\usepackage[left=2.6cm, right=2.6cm, top=1.5cm, includehead, includefoot]{geometry}
\usepackage[dvipsnames]{xcolor}
\usepackage[d]{esvect}
\usepackage{listings}
\usepackage{enumitem} % To allow for alph in enumerate
\usepackage{braket}
\usepackage{float} % To allow for H setting in figures.

%% header
\pagestyle{fancy}
\fancyhead[L]{\bf\large CSC236 UTM \\ Assignment 1}
\fancyhead[R]{\bf\large Fall 2024 \\Due Sept 30}
%\fancyfoot[C]{Page \thepage\ of 2}
\setlength{\headheight}{35pt}

\begin{document}
    \noindent\textbf{Question 1.} Define the predicates
    \[
        P(n)\colon \text{ For any set } A\text{, if } |A|=n \text{ then } |\mathcal{P}(A)|=2^n
    \]
    \[
        Q(A,n)\colon |A|=n \Longrightarrow |\mathcal{P}(A)|=2^n
    \]
    \begin{enumerate}[label=\alph*)]
        \item Prove \(\forall n \in \mathbb{N} , P(n)\).
        \begin{proof}
            \noindent\textbf{Base Case.} To show \(P(0)\), consider any set A such that \(\left\vert A \right\vert = 0\). Then \(A = \varnothing \) and its only subset is \(\varnothing\). Thus \(\mathcal{P} (A) = \{\varnothing\} \implies \left\vert \mathcal{P}(A) \right\vert = 1 = 2^0\), verifying that \(P(0)\) is true.

            \noindent\textbf{Induction Step.} Suppose \(P(k)\) holds for some \(k \in \mathbb{N} \). Now \(P(k+1)\) will be proven to hold. Let \(A\) be a set such that \(\left\vert A \right\vert = k+1\). \(k+1\) is at least \(1\), so \(A\) possesses at least one element, which will be denoted as \(a\).
            
            Consider the set \(A\setminus \{ a \} \). Since \(\bigl\vert A\setminus\{a\}\bigr\vert = k \), by the induction hypothesis,
            \[
                \bigl\vert\mathcal{P}(A\setminus\{a\})\bigr\vert = 2^k
            \]
            Notice that \(\mathcal{P}(A\setminus\{a\})\) contains all the subsets of \(A\) that do not contain \(a\). The remaining subsets must all contain \(a\). The remaining subsets of \(A\) can be obtained by taking every individual element in \(\mathcal{P}(A\setminus\{a\})\) and unioning it with \(\{a\}\). Thus \(A\) contains twice as many subsets as \(A\setminus \{a\}\). In mathematical terms, 
            \[
                \left\vert \mathcal{P}(A) \right\vert = 2 \cdot \left\vert \mathcal{P}(A\setminus\{a\}) \right\vert = 2 \cdot 2^k = 2^{k+1} 
            \]
            It has been shown that \(P(k+1)\) holds.

            \noindent By the principle of simple induction, \(\forall n \in \mathbb{N} , P(n)\). 

        \end{proof}
        \item  Prove that for every set \(A\), \(\forall n \in \mathbb{N} \text{, } Q(n)\).
        This method does not work. Here is the attempt at the proof:
        \begin{proof}
            Fix a set \(A\). Proceed with using simple induction.

            \textbf{Base Case.} Let \(n=0\). To show \(Q(A, n)\) holds, suppose that \(\left\vert A \right\vert = 0\). Then \(A = \varnothing \). Thus \(\mathcal{P} (A) = \{ \varnothing  \} \implies \left\vert \mathcal{P} (A) \right\vert = 1 = 2^0 \).
            
            Thus \(Q(A, 0)\).

            \textbf{Induction Step.} Suppose that \(Q(A, k)\) holds for some \(k \in \mathbb{N}\). To show \(Q(A, k+1)\), suppose \(\left\vert A \right\vert = k+1\). However, this is where the problem arises. 

            The induction hypothesis cannot be utilised since our assumption requires \(\left\vert A \right\vert = k+1\), while the condition to use the induction hypothesis is \(\left\vert A \right\vert = k\).
            
            Thus the proof by induction cannot be continued.

        \end{proof}
    \end{enumerate}
    \pagebreak
    \noindent\textbf{Question 2.} Let \(n,m \in \mathbb{N}\). Let \(A, B\) be arbitrary finite sets of size \(m\) and \(n\) respectively.
    \begin{enumerate}[label=\alph*)]
        \item How many fuctions are there with domain \(A\) and co-domain \(B\)?
        It can be shown using simple induction on \(m\) that the answer to this question is \(n^m\).
        \begin{proof}
            First, particular edge cases will be examined.
            For \(n,m \in \mathbb{N} \), define the predicate
            \[
                P(m) \colon \text{ For every positive natural } n \text{, there are } n^m \text{ functions with finite domain of size } m
            \]
            \[
                \text{ and finite co-domain of size } n
            \]
            Fix \(m \in \mathbb{N}\).

            \textbf{Base Case.} Let \(m = 0\). There are no functions that can map to nothing, therefore the number of functions is \(0^n = 0\).

            \textbf{Induction Step.} Suppose that \(P(n, k)\) holds for every \(n \in \mathbb{N}\), but only for some \(k \in \mathbb{N}\). Let \(A, B\) be finite sets such that \(\left\vert A \right\vert = k+1\) and \(\left\vert B \right\vert = n\). \(A\) contains at least one element \(a\). Consider the set \(A\setminus\{a\}\). By the induction hypothesis, there are \(n^k\) functions with domain \(A\setminus\{a\}\) and co-domain \(B\). For every such function \(f_k\) and some \(b \in B\) , define a new function
            \[
                f_b(x) = \begin{dcases}
                    f_k(x), &\text{ if } x \in A\setminus \{a\} ;\\
                    b, &\text{ if } x=a;
                \end{dcases}
            \]
            Every function that maps elements from \(A\) to \(B\) can be written in this form. Thus there are \(n^k*n = n^{k+1} \) fuctions that map from \(A\) to \(B\).

        \end{proof}

        \item Use part (a) to prove the original statement in Q1 directly without the use of induction.
        \begin{proof}
            Let \(A\) be a set such that \(|A| = n\). Every subset \(A^\prime\) of \(A\) can be defined as a function \(f\) that maps elements of \(A\) to \(\{0,1\}\):
            \[
                \text{For any } a \in A \text{, if } a \in A^\prime, f(a) = 1 \text{. Otherwise, } f(a) = 0
            \]
            From the previous part, there are \(2^n\) different functions with domain \(A\) and co-domain \(\{0,1\}\), which also means that there are \(2^n\) subsets of \(A\), which implies that \(\mathcal{P} (A)=2^n\).

        \end{proof}
    \end{enumerate}
    \pagebreak
    \noindent\textbf{Question 3.} In propositional logic, you have seen the connectives $\neg, \wedge, \vee, \rightarrow, $ and $\leftrightarrow$. Prove using structural induction that any proposition built using these connectives is equivalent to a proposition built only using $\neg, \rightarrow$.
    \begin{proof}
        The proof will be done using structural induction.

        For a proposition \(P\), Define the predicate
        \[
            Q(P) : P \text{ is equivalent to some proposition built only using } \neg \text{, } \rightarrow
        \]
        \textbf{Base Case.} For all \(P_i(x_{j_1}, x_{j_2}, \dots, x_{j_k})\), they are equivalent to a proposition built only using \(\neg, \rightarrow\), which are themselves. Thus \(Q(P_i(x_{j_1}, x_{j_2}, \dots, x_{j_k}))\) holds.

        \textbf{Induction Step.} Let \(A,B\) be propositions such that \(Q(A)\) and \(Q(B)\) hold. It follows that \(A\equiv C, B\equiv D\), where \(C, D\) are propositions built from only \(\neg \text{, } \rightarrow\). 5 recursive cases will be considered.

        \begin{enumerate}
            \item \(\neg A \equiv \neg C\), thus \(Q(\neg A)\) holds
            \item \(A \implies B \equiv C \implies D\), \(Q(A \implies B)\) holds
            \item \(A \land B \equiv C \land D\). It will be shown using a truth table that \(C \land D \equiv \neg(C\implies \neg D)\).
            \begin{displaymath}
                \begin{array}{c c | c | c}
                    C & D & C \land D & \neg(C\implies \neg D)\\
                    \hline
                    T & T & T & T\\
                    T & F & F & F\\
                    F & T & F & F\\
                    F & F & F & F\\
                \end{array}
            \end{displaymath}
            Since \(A \land B \equiv \neg(C\implies \neg D)\), which is a proposition built from only \(\neg, \implies\). Thus \(Q(A \land B)\) holds.

            \item \(A \lor B \equiv C \lor D\). It will be shown using a truth table that \(C \lor D \equiv \neg(\neg C\implies D)\).
            \begin{displaymath}
                \begin{array}{c c | c | c}
                    C & D & C \lor D & \neg(\neg C\implies D)\\
                    \hline
                    T & T & T & T\\
                    T & F & T & T\\
                    F & T & T & T\\
                    F & F & F & F\\
                \end{array}
            \end{displaymath}
            Since \(A \lor B \equiv \neg(\neg C\implies D)\), which is a proposition built from only \(\neg, \implies\). Thus \(Q(A \lor B)\) holds.

            \item \(A \iff B \equiv C \iff D\). It will be shown using a truth table that \(C \iff D \equiv \neg((C \implies D) \implies \neg(D \implies C))\).
            \begin{displaymath}
                \begin{array}{c c | c | c}
                    C & D & C \iff D & \neg((C \implies D) \implies \neg(D \implies C))\\
                    \hline
                    T & T & T & T\\
                    T & F & F & F\\
                    F & T & F & F\\
                    F & F & T & T\\
                \end{array}
            \end{displaymath}
            Since \(C \iff D \equiv \neg((C \implies D) \implies \neg(D \implies C))\), which is a proposition built from only \(\neg, \implies\). Thus \(Q(A \iff B)\) holds.
        \end{enumerate}
        By the principle of structural induction, \(Q(P)\) holds for all propositions \(P\).
        
    \end{proof}
    \pagebreak
    \noindent\textbf{Question 4.} Consider the following two-pointer style Python program which finds whether a given string $s$ is a palindrome or not:
    \begin{lstlisting}[language=Python]
    def check_if_palindrome(s):
        left = 0
        right = len(s) - 1
        while left < right:
            if s[left] != s[right]:
                return False
            left += 1
            right -= 1
        return True
    \end{lstlisting}
    Prove correctness and termination. Clearly state the loop variant and the loopinvariant, and use induction properly.
\end{document}