\documentclass{article}
\usepackage[margin=1.0in]{geometry}
\usepackage{amssymb,amsmath,amsthm,amsfonts}
\usepackage{enumitem}
\usepackage{xcolor}

\newcommand{\Z}{\mathbf{Z}}
\newcommand{\N}{\mathbf{N}}
\newcommand{\R}{\mathbf{R}}
\newcommand{\Q}{\mathbf{Q}}
\newcommand{\C}{\mathbf{C}}

\newcommand{\id}{\mathrm{id}}
\newcommand{\op}{\mathrm{op}}
\newcommand{\diam}{\mathrm{diam}}
\newcommand{\Tr}{\mathrm{Tr}}
\newcommand{\im}{\mathrm{im}}
\newcommand{\rank}{\mathrm{rank}}

\newcommand{\cl}[1]{\overline{#1}}

\swapnumbers % places numbers before thm names

\theoremstyle{plain} % The "plain" style italicizes all body text.
	\newtheorem{thm}{Theorem}
		\numberwithin{thm}{section} % Theorem numbers are determined by section.
	\newtheorem{lemma}[thm]{Lemma}
	\newtheorem{prop}[thm]{Proposition}
	\newtheorem{cor}[thm]{Corollary}

\theoremstyle{definition}
    \newtheorem{defn}[thm]{Definition}
	\newtheorem{example}[thm]{Example}
	\newtheorem{exercise}[thm]{Exercise} %Exercise

\begin{document}
    \section{MAT301 Notes: Week 2 (Fall 2024)}
    \begin{example}
        Let \(G\) be a group. Prove that
        \[
            C_G (a) = \forall a \in G \iff G \text{ is Abelian}
        \]
        Note that the centralizer \(C_G(a)\) is defined by \(C_G(a) = \{ g \in G : ga=ag  \} \)
        \begin{proof}
            We begin by proving the \(\impliedby\) direction.

            Let \(a \in G\). Then, since \(G\) is Abelian, every element commutes with \(a\). Therefore

            \[
                C_G(a) = G
            \]  

            Conversely, let \(x,y \in G\). Then we have
            \[
                x \in C_G(x)=G=C_G(y) \implies xy=yx
            \]

        \end{proof}
    \end{example}
    \begin{example}
        Let \(G\) be a group. Prove that \(G\) is Abelian if and only if \((xy)^2=x^2y^2 , x,y \in G\).
        \begin{proof}
            Suppose \(G\) is Abelian. Then
            \[
                (xy)^2=(xy)(xy)=(xx)(yy)=x^2 y^2
            \] 
            Next suppose \((xy)^2 = x^2 y^2\). Then
            \[
                xxyy = x^2 y^2 = (xy)^2 = xyxy
            \]
            Applying \(x^{-1}\) to the left and \(y^{-1}\) to the right obtains
            \[
                x^{-1} xxyy y^{-1} = x^{-1} xyxy y^{-1} \implies xy=yx
            \]
        \end{proof} 
    \end{example}
    \begin{exercise}
        Let \(G\) be a group. Let \(a, b \in G\), where \(a\) has odd order. Prove that if \(a = bab\), then \(b = b^{-1} \). 
        \begin{proof}
            Suppose that \(a = bab\). Then if we apply \(ab\) to the right on both sides, we have
            \[
                a(ab) = bab(ab) \implies a^2 b = ba^2 \tag{1}
            \]
            Since \(a\) is odd order, \(a^{2n+1} = e \implies a = a^{2n+2} \), where \(n \in \mathbb{N} \). Thus
            \[
                a = bab = ba^{2n+2}b = b(a^2)^{n+1}b
            \] 
            We apply (1) \(n+1\) times to obtain that
            \[
                a = b(a^2)^{n+1}b = b^2 (a^2)^{n+1} = b^2 a^{2n+2} = b^2 a \implies e = b^2 \implies b = b^{-1} 
            \]
            We are done.
            
        \end{proof} 
    \end{exercise}
\end{document}